\documentclass[UTF8]{article}
\usepackage{color}
\usepackage{amsmath}
\usepackage{CTEX}
\usepackage{graphicx}
\usepackage{subcaption}

\definecolor{dkgreen}{rgb}{0,0.6,0}
\definecolor{gray}{rgb}{0.5,0.5,0.5}
\definecolor{mauve}{rgb}{0.58,0,0.82}

\begin{document}
	利用本章节学习的U统计量构造一个统计分析案例,可以参照如下结构:
	\section{统计模型及具体问题}
	在现实生活生经常会遇到这样的问题:金融行业的收入是否是传统行业的两倍,上海地区的人均收入是否是西藏地区的五倍。
	
	将这些问题抽象为统计问题,也就是比较两个总体$(F,G)$的均值$(\mu_1,\mu_2)$的比值$\frac{\mu_1}{\mu_2}$是否是一个给定的值$c$,这是一个典型的假设检验的问题:
	
	$$H_0:\frac{\mu_1}{\mu_2}=c \qquad vs\qquad H_1:\frac{\mu_1}{\mu_2}\neq c$$
	
	本次作业将使用$U$统计量构造该假设检验的统计量并进行理论的推到和数据的模拟。
	\section{统计方法及理论}
	设$X_1,...,X_{n1}$为独立同分布于总体$F$的样本,同理$Y_1,...,Y_{n2}$为独立同分布于总体$G$的样本,即:
	$$X_1,\cdots,X_{n1}\quad i.i.d \sim F$$
	$$Y_1,\cdots,Y_{n2}\quad i.i.d \sim G$$
	对于假设检验问题
	$$H_0:\frac{\mu_1}{\mu_2}=c \qquad vs\qquad H_1:\frac{\mu_1}{\mu_2}\neq c$$
	构造核函数
	$$\phi(X_i, Y_j)=\frac{X_i}{Y_j}$$
	易知
	$$E[\phi(X_i, Y_j)]=E[\frac{X_i}{Y_j}]=\frac{E[X_i]}{E[Y_j]}=\frac{\mu_1}{\mu_2}$$
	因此$\phi(X_i, Y_j)=\frac{X_i}{Y_j}$是$\frac{\mu_1}{\mu_2}$的一个无偏估计,阶为$(1, 1)$,根据此核函数构造$U$统计量为
		$$U_{n_1,n_2}=\frac{1}{n_1n_2}\sum_{i=1}^{n_1}\sum_{j=1}^{n_2}\phi(X_i, Y_j)=\frac{1}{n_1n_2}\sum_{i=1}^{n_1}\sum_{j=1}^{n_2}\frac{X_i}{Y_j}$$
	定义
	$$\sigma^2_{10} := Cov(\phi(X_1,Y_1),\phi(X_1,Y^*_1))=Cov(\frac{X_1}{Y_1},\frac{X_1}{Y^*_1})
	=E[\frac{{X_1}^2}{Y_1Y^*_1}]-\frac{{\mu_1}^2}{{\mu_2}^2}=\frac{Var(X_1)}{{\mu_2}^2}$$
	$$\sigma^2_{01} := Cov(\phi(X_1,Y_1),\phi(X^*_1,Y_1))=E[\frac{X_1X^*_1}{{Y_1}^2}]-\frac{{\mu_1}^2}{{\mu_2}^2}
	={\mu_1}^2(\frac1{E[{Y_1}^2]}-\frac1{{\mu_2}^2})$$
	则根据$U$统计量的性质有
	$$\sqrt{n_1+n_2}(U_{n_1,n_2}-\frac{\mu_1}{\mu_2})\rightarrow N(0,\sigma^2)$$
	其中
	$$\sigma^2=\frac{{m_1}^2}{p}\sigma^2_{10}+\frac{{m_2}^2}{1-p}\sigma^2_{01}
	=\frac1{p}\sigma^2_{10}+\frac1{1-p}\sigma^2_{01}, \frac{n_1}{n_1+n_2}\rightarrow p$$
	为此需要构造$\sigma^2_{10}$和$\sigma^2_{01}$的渐近估计。
	
	根据$Var(X_1)$的估计为$\phi(X_1, X_2)=\frac1{2}(X_1-X_2)^2$可得,$\sigma^2_{10}$的一个$U$统计量核函数为
	$$\phi(X_1, X_2,Y_1,Y_2) = \frac{(X_1-X_2)^2}{2Y_1Y_2}$$
	同理可得$\sigma^2_{01}$的一个$U$统计量核函数为
	$$\phi(X_1,Y_1,Y_2) = X_1X_2(\frac1{})$$
	\section{数据模拟}:在构造的数据上对所用方法进行模拟,展示出所用方法的有
	效性以及优缺点分析。如果有可能,可以结合经典的方法,把新的方法
	和以前放到一起进行比较。
	\section{实际数据分析}:把方法应用到一个具体的实际数据上,展示方法的实际
	效果。
\end{document}



